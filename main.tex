\documentclass[a4paper,10pt]{article}
\usepackage[a4paper,margin=0.75in]{geometry}
\usepackage{amsmath}
\usepackage{amsfonts}
\usepackage{amssymb}
\usepackage{subfiles}
\usepackage{tcolorbox}
\usepackage{siunitx}
\usepackage[straightvoltages]{circuitikz}
\usepackage{pgfplots}
\usepackage{xifthen}
\usepackage{float}
\usepackage{xcolor}
\usepackage{multicol}
\usepackage{tabularx}
\usepackage{parskip}
\usepackage{unicode-math}
\usepackage{hyperref}
\usepackage{framed}
\usepackage[version=4]{mhchem}

\setmainfont{STIX Two Text}
\setmathfont{STIX Two Math}

\tcbuselibrary{skins}
\tcbuselibrary{breakable}

\tcbset{ignore nobreak}

\usetikzlibrary{arrows}
\usetikzlibrary{calc}
\usetikzlibrary{decorations.pathmorphing}
\usetikzlibrary{shapes}
\usetikzlibrary{positioning}
\usetikzlibrary{decorations.pathreplacing}
\usetikzlibrary{patterns}
\usetikzlibrary{fpu}

\setcounter{tocdepth}{1}

\pgfplotsset{width=0.85\linewidth,compat=1.5.1}
\usepgfplotslibrary{fillbetween}

\title{`O'-Level Physics Notes}
\author{Ng Kang Zhe}
\date{2018}

\newtcolorbox{preamb}{
	breakable=false,
	top = 0.5mm, bottom = 0.5mm,
	left = 0.5mm, right = 0.5mm,
	colback = white,
	title=Preamble,
}

\newtcolorbox[auto counter,number within=subsection]{physdef}[2][]{
	sharp corners=all,
	breakable=false,
	top = 0.5mm, bottom = 0.5mm,
	left = 0.5mm, right = 0.5mm,
	colback = white,
	title=Definition~\thetcbcounter: #2,
	#1,
}

\newcommand\pdef[2]{\begin{physdef}{#1} #2 \end{physdef}}

\newtcolorbox[auto counter,number within=subsection]{physeqn}[2][]{
	sharp corners=all,
	breakable=false,
	top = 0.5mm, bottom = 0.5mm,
	left = 0.5mm, right = 0.5mm,
	colback = white,
	title=Equation~\thetcbcounter: #2,
	#1,
}

\newcommand\peqn[4][]{
	\begin{physeqn}{#2}
		#3
		\begin{equation*}
		\ifthenelse{\equal{#1}{}}{}{\begin{#1}}
		#4
		\ifthenelse{\equal{#1}{}}{}{\end{#1}}
		\end{equation*}
		
	\end{physeqn}
}

\begin{document}

\maketitle

\begin{abstract}
	These notes are for the `O'-Level Physics syllabus [6091]. They might be similar in content to other physics courses too, though structured differently.
	
	These notes are written by myself, which means they are prone to typos and errors. If you find errata, do contact me so I can remedy. or give you access to the Github repository for you to push any changes.
	
	Some code (especially the \texttt{tcolorboxes}) are copied fron \href{https://github.com/4yn/a-lv-notes}{\color{blue} \underline{4yn's a-lv-notes repository\footnote{https://github.com/4yn/a-lv-notes}}}.
	
	Do whatever you want with these notes. Reproduce them, distribute them, use material from them, go crazy. I don't mind. Unless you republish it without any changes under your own name, we won't have a problem.
	
	Use these notes with caution.
\end{abstract}

\begin{multicols}{2}
\tableofcontents
\end{multicols}

\vspace*{\fill}
\hrule
\begin{center}
	This document consists of \textbf{26} pages including the cover page.
\end{center}

\newpage
\begin{multicols}{2}
	
\part{Measurement}
\subfile{content/chapter01}
\part{Newtonian Mechanics}
\subfile{content/chapter02}
\subfile{content/chapter03}
\subfile{content/chapter04}
\subfile{content/chapter05}
\subfile{content/chapter06}
\subfile{content/chapter07}
\part{Thermal Physics}
\subfile{content/chapter08}
\subfile{content/chapter09}
\subfile{content/chapter10}
\subfile{content/chapter11}
\part{Waves}
\subfile{content/chapter12}
\subfile{content/chapter13}
\subfile{content/chapter14}
\subfile{content/chapter15}
\part{Electricity and Magnetism}
\subfile{content/chapter16}
\subfile{content/chapter17}
\subfile{content/chapter18}
\subfile{content/chapter19}
\subfile{content/chapter20}
\subfile{content/chapter21}
\subfile{content/chapter22}
\end{multicols}
\vspace{3em}
\begin{center}
{\Large \textbf{End of Document}}

\textit{Have fun studying and all the best for your examinations!}

\href{https://github.com/kangzhe3067/physicsNotes}{\color{gray} \texttt{https://github.com/kangzhe3067/physicsNotes}}
\end{center}

\end{document}
