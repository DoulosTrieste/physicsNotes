\documentclass[a4paper,10pt]{article}

\usepackage[margin=2cm]{geometry}
\usepackage{amsmath}
\usepackage{amsfonts}
\usepackage{amssymb}
\usepackage{subfiles}
\usepackage{tcolorbox}
\usepackage{siunitx}
\usepackage{circuitikz}
\usepackage{pgfplots}
\usepackage{cancel}
\usepackage{xifthen}
\usepackage{float}
\usepackage{xcolor}
\usepackage{multicol}
\usepackage{tabularx}
\usepackage{lmodern}

\renewcommand{\familydefault}{\sfdefault}

\tcbuselibrary{skins}
\tcbuselibrary{breakable}

\pgfplotsset{width=0.75\linewidth,compat=1.5.1}
\usetikzlibrary{arrows}
\usepgfplotslibrary{fillbetween}

\title{Physics Notes}
\author{Ng Kang Zhe}

\setlength{\parindent}{0em}

\newtcolorbox{preamb}{
	breakable=true,
	top = 0.5mm, bottom = 0.5mm,
	left = 0.5mm, right = 0.5mm,
	colback = white,
	title=Preamble,
}

\newtcolorbox[auto counter,number within=section]{physdef}[2][]{
	sharp corners=all,
	breakable=true,
	top = 0.5mm, bottom = 0.5mm,
	left = 0.5mm, right = 0.5mm,
	colback = white,
	title=Definition~\thetcbcounter: #2,
	#1,
}

\newcommand\pdef[2]{\begin{physdef}{#1} #2 \end{physdef}}

\newtcolorbox[auto counter,number within=section]{physeqn}[2][]{
	sharp corners=all,
	breakable=true,
	top = 0.5mm, bottom = 0.5mm,
	left = 0.5mm, right = 0.5mm,
	colback = white,
	title=Equation~\thetcbcounter: #2,
	#1,
}

\newcommand\peqn[4][]{
	\begin{physeqn}{#2}
		#3
		\begin{equation*}
		\ifthenelse{\equal{#1}{}}{}{\begin{#1}}
		#4
		\ifthenelse{\equal{#1}{}}{}{\end{#1}}
		\end{equation*}
		
	\end{physeqn}
}

\begin{document}

\maketitle

\begin{abstract}
	These notes are written by myself, which means they are prone to typos and errors. If you find errata, do contact me so I can remedy. or give you access to the Github repository for you to push any changes.
	
	Use these notes with caution.
\end{abstract}

\newpage
{\fontfamily{lmss}\selectfont
\begin{multicols*}{2}
\part{Measurement}
\subfile{content/chapter01}

\part{Newtonian Mechanics}
\subfile{content/chapter02}
\subfile{content/chapter03}
\subfile{content/chapter04}
\subfile{content/chapter05}
\subfile{content/chapter06}
\subfile{content/chapter07}

\part{Thermal Physics}
\subfile{content/chapter08}
\subfile{content/chapter09}
\subfile{content/chapter10}
\subfile{content/chapter11}

\part{Waves}
\subfile{content/chapter12}
\subfile{content/chapter13}
\end{multicols*}
}

\end{document}
