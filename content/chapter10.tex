\documentclass[../main.tex]{subfiles}

\begin{document}
	\section{Thermal Properties of Matter}	
	\begin{preamb}
		Matter has some properties when it comes to heat. These preambles are also getting difficult to write because I'm running out of ideas.
	\end{preamb}
	
	\pdef{Heat Capacity}{Heat capacity \( C \) is the amount of heat energy required to raise the temperature of an object by \SI{1}{\kelvin}. Its relationship can be expressed as \[ \Delta Q = C \Delta T\] The SI unit of heat capacity is joule per kelvin [\si{\joule \per \kelvin}].}
	
	\pdef{Specific Heat Capacity}{Specific Heat capacity \( c \) is the amount of heat energy required to raise the temperature of a unit mass of an object by \SI{1}{\kelvin}. Its relationship can be expressed as \[ \Delta Q = mc \Delta T\] The SI unit of heat capacity is joule per kelvin per kilogram [\si{\joule \per \kelvin \per \kilo\gram }].}
	
	\pdef{Latent Heat}{Latent heat is the amount of heat energy required to allow a unit mass of an object to transition from one state from another. In general, \[ Q_{f/v} = ml_{f/v} \] where \(l_{f/v}\) is the specific latent heat of fusion/vaporisation, the heat energy required to melt or freeze/vaporise or condense a unit mass. The SI unit of specific latent heat is joule per kilogram [\si{\joule \per \kilo\gram}].}
\end{document}
