\documentclass[../main.tex]{subfiles}

\begin{document}
	\section{Temperature}
	\begin{preamb}
		In this chapter we will learn how to make a thermometer because you can't buy one in practical exam.
	\end{preamb}
	
	\pdef{Heat}{Heat is the amount of thermal energy that is being transferred from a hotter to a colder region.}
	
	\pdef{Ice Point}{The ice point is the temperature of pure melting ice at one atmosphere, and is assigned a value of \SI{0}{\celsius}.}
	
	\pdef{Steam Point}{The steam point is the temperature of pure melting ice at one atmosphere, and is assigned a value of \SI{100}{\celsius}.}
	
	\pdef{Thermometric Property}{A thermometric property is a property of matter that varies continuously with temperature.}
	Some examples of this include the volume of an object, the electromotive force of a thermocouple, and the height of a liquid column.
	
	\peqn{Thermometry Formula}{To make a thermometer, you need some thermometric property \(X\) at temperatures \SI{0}{\celsius}, \SI{100}{\celsius}, and some temperature \(\theta\) \,\si{\celsius}. Then you plug them into this formula}{\theta \, \si{\celsius} = \frac{X_\theta - X_0}{X_{100} - X_0} \times \SI{100}{\celsius}}
	
	\peqn{Temperature Conversion}{To convert from degrees celsius [\si{\celsius}] to kelvin [\si{\kelvin}],}{[\si{\kelvin}] = [\si{\celsius}]+273.15}
\end{document}
