\documentclass[../main.tex]{subfiles}

\begin{document}
	\section{Kinetic Model of Matter}
	
	\begin{preamb}
		Matter is made up of small particles that behave in certain ways under different conditions. In this chapter we will accurately describe the particulate nature of matter and how it behaves under different temperature and pressure conditions.
	\end{preamb}
	
	\subsection{Three States of Matter}
	\begin{center}
		\begin{tabularx}{0.85\linewidth}{Xccc}
			\hline \hline
			Property & Solid & Liquid & Gas \\ 
			\hline
			Shape & fixed & not fixed & not fixed \\ 
			Volume & fixed & fixed & not fixed \\ 
			Compressible? & no & no & yes \\ 
			\hline 
		\end{tabularx} 
	\end{center}
	
	When prompted to describe a state, you might want to talk about its:
	\begin{itemize}
		\item arrangement of particles
		\item forces between particles
		\item kinetic energy of particles
		\item motion of particles
	\end{itemize}
	as written like in the next few subsections.
	
	\subsubsection{Solids}
	Solids are
	\begin{itemize}
		\item closely packed in an orderly manner
		\item held together by strong forces of attraction
		\item have enough energy to only vibrate and rotate about their fixed positions
		\item cannot move around freely
	\end{itemize}
	
	\subsubsection{Liquids}
	Liquids are 
	\begin{itemize}
		\item arranged in a disorderly manner
		\item have weaker forces of attraction than the particles of a solid
		\item have more kinetic energy than particles of the substance in the solid state, and are not held in fixed positions
		\item can move freely throughout the liquid
	\end{itemize}
	
	\subsubsection{Gases}
	Gases are
	\begin{itemize}
		\item spread far apart from one another
		\item have weaker forces of attraction than the particles of a liquid
		\item have a lot of kinetic energy and are not held in fixed positions
		\item can move about rapidly in any direction
	\end{itemize}
	
	\pdef{Brownian Motion}{Particles are in constant random motion. Brownian motion arises due to these random motions of particles in a fluid.}
	
	\subsection{Gas Laws}
	There are three gas laws. 
	\pdef{Ideal Gas Law}{As a result of the three gas laws to be presented below, the relationship for an ideal gas between its temperature, pressure, and volume can be expressed as \[pV = nRT\] where \(nR\) is some constant.}
	
	\peqn{Charles Law}{Charles Law states that the pressure of a gas is directly proportional to its temperature \textit{if the volume stays constant (isochoric)}. Mathematically,}{p \propto T}
	
		\begin{center}
			\begin{tikzpicture}
			\pgfplotsset{ticks=none}
			\begin{axis}[
			axis lines = middle,
			xlabel = \(T\),
			ylabel = \(p\)
			]
			\addplot [domain=0:10] {x};
			\end{axis}
			\end{tikzpicture}
		\end{center}
	
	
	\peqn{Boyle's Law}{Boyle's law states that the pressure of a gas is inversely proportional to the volume of the gas \textit{if the temperature stays constant (isothermic)}. Mathematically,}{p \propto \frac{1}{V}}
	
		\begin{center}
			\begin{tikzpicture}
			\pgfplotsset{ticks=none}
			\begin{axis}[
			axis lines = middle,
			xlabel = \(V\),
			ylabel = \(p\),
			ymax = 10,
			xmin = 0,
			ymin = 0,
			samples = 100
			]
			\addplot [domain=0:1] {1/x};
			\end{axis}
			\end{tikzpicture}
		\end{center}
		
	\peqn{Gay-Lussac's Law}{Gay-Lussac's Law states that the volume of a gas is directly proportional to its temperature \textit{if the pressure stays constant (isobaric)}. Mathematically,}{V \propto T}
	
		\begin{center}
			\begin{tikzpicture}
			\pgfplotsset{ticks=none}
			\begin{axis}[
			axis lines = middle,
			xlabel = \(T\),
			ylabel = \(V\)
			]
			\addplot [domain=0:10] {x};
			\end{axis}
			\end{tikzpicture}
		\end{center}

	\peqn{Avogadro's Law}{(\textit{This is not in this syllabus but it is in O-Level Chemistry so I'll put it here.}) Avogadro's law states that the amount of gas is directly proportional to the volume of the gas. Mathematically,}{n \propto V}
\end{document}