	\documentclass[../main.tex]{subfiles}

\begin{document}
	\section{Energy, Work, and Power}
	\begin{preamb}
		The study of energy and matter form the basis of physics. In this chapter we will look at the concept of energy, work done, power, and other relevant quantities.
	\end{preamb}
	
	\subsection{Energy}
	\pdef{Kinetic Energy}{Kinetic energy is the energy an object possesses when it is moving. It is given as \[E_K = \frac{1}{2} m v^2\] The SI unit of kinetic energy is the joule [\si{\joule}].}
	
	\pdef{Gravitational Potential Energy}{Gravitational potential energy is defined as how much work can be done by the gravitational force from a height \(h\) away. It is given as \[E_P = mgh\] The SI unit of gravitational potential energy is the joule [\si{\joule}]. }
	
	\peqn{Efficiency}{Efficiency is calculated by }{\eta = \frac{\mathrm{output}}{\mathrm{input}} \times 100 \%}
	
	\pdef{First Law of Thermodynamics}{Energy is conserved in a closed system, i.e. \[\Delta E_T = 0\]}
	
	\pdef{Work Done}{The work done by a force is the energy transferred by a force to an object. It is given as the numerical product of the force and the displacement in the direction of the force \[W = Fs\] The SI unit of work done is the joule [\si{\joule}].}
	
	\pdef{Power}{Power is defined as the rate of work done. It is calculated as \[P = \frac{W}{t}\] The SI unit of power is the watt [\si{\watt}].}
\end{document}