\documentclass[../main.tex]{subfiles}

\begin{document}
	\section{Kinetic Model of Matter}
	
	\begin{preamb}
		Matter is made up of small particles that behave in certain ways under different conditions. In this chapter we will accurately describe the particulate nature of matter and how it behaves under different temperature and pressure conditions.
	\end{preamb}
	
	\subsection{Three States of Matter}
	The three most common states of matter are solid, liquid, and gas. 
	
	\textbf{Solids} have a fixed shape, and have a fixed volume. They have strong forces of attraction between particles. The particles vibrate around a fixed point in the solid.
	
	\textbf{Liquids} have a shape that follows the container, and have a fixed volume. They have slightly weaker forces of attraction between particles compared to solids. The particles flow and slide past each other within the liquid.
	
	\textbf{Gases} have do not have shape, and do not have a fixed volume. They have very weak forces of attraction between particles. The particles move freely.
	
	\pdef{Brownian Motion}{Particles are in constant random motion. Brownian motion arises due to these random motions of particles in a fluid.}
	
\end{document}