\documentclass[../main.tex]{subfiles}

\begin{document}
	\section{Mass, Weight, and Density}
	\begin{preamb}
		Matter is anything that takes up space and has mass. The three quantities we are exploring today will allow us to describe matter in different ways.
	\end{preamb}
	
	\subsection{Mass}
		\pdef{Mass}{Mass is the \textbf{amount of matter} in a body. The SI unit of mass is the kilogram [\si{\kilo\gram}].}
		The magnitude of mass depends on the number of atoms in the body. 
		
		Mass is a \textbf{scalar} quantity. It can be measured with an \textbf{electronic mass balance}.
		
		\pdef{Inertia}{The inertia of an object refers to the reluctance of the object to change its state of rest or motion, due to its mass.}
		
	\subsection{Weight}
		\pdef{Weight}{The weight of an object is defined as the \textbf{gravitational force acting on it} due to gravity. The weight of an object \(w\) with mass \(m\) is equal to \[w = mg\] where \(g\) is the local gravitational field strength. The SI unit of weight is the newton [\si{\newton}].}
		Weight is a force, therefore it is a \textbf{vector} quantity. It can be measured with a \textbf{spring balance}.
		
		\pdef{Gravitational Field}{A gravitational field is a region in which a mass experiences a force due to gravitational attraction. The gravitational field strength is the gravitational force acting per unit mass. On Earth, is equal to \[g = \SI{10}{\meter \per \second \squared} = \SI{10}{\newton \per \kilo \gram}\]}
		
	\subsection{Density}
		\pdef{Density}{The density of an object is its mass per unit volume. The density of an object \(\rho\) with mass \(m\) and volume \(V\) is equal to \[\rho = \frac{m}{V}\] The SI unit of density is kilogram per cubic metre [\si{\kilo\gram\per\cubic\meter}].}
		When an object is placed in a liquid,
		\[ 
			\text{the object will}\begin{cases}
				\text{float} & \rho_{\mathrm{object}} < \rho_{\mathrm{liquid}} \\
				\text{suspend} & \rho_{\mathrm{object}} = \rho_{\mathrm{liquid}} \\
				\text{sink} & \rho_{\mathrm{object}} > \rho_{\mathrm{liquid}} \\
			\end{cases}			
		\]
		
\end{document}