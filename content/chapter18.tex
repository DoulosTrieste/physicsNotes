\documentclass[../main.tex]{subfiles}

\begin{document}
	\section{DC Circuits}
	\begin{preamb}
		Most things at our homes run on direct current (DC). In this chapter we will explore how DC circuits behave and how it is used to make the many circuits and electronic devices around us.
	\end{preamb}
	
	\peqn{Kirchhoff's Current Law}{\textit{(This isn't in syllabus.)} The current flowing in a junction must equal to the current flowing out of a junction.}{\Sigma I_{\mathrm{node}} = 0}
	\peqn{Kirchhoff's Voltage Law}{\textit{(This isn't in syllabus.)} The algebraic sum of voltages in a loop/mesh is equal to zero.}{\Sigma V_{\mathrm{mesh}} = 0}
	
	\subsection{Series Circuits}
	We will look at this series circuit for this subsection.
	\begin{center}
		\begin{circuitikz}
			\draw (2,2) to[battery1, i^=\(I_1\), v=\(V_1\)] (4,2) -- (6,2) -- (6,0) -- (5,0) to[european resistor, i<=\(I_3\), v=\(V_3\)] (3,0) to[european resistor, i<=\(I_2\), v=\(V_2\)] (1,0) -- (0,0) -- (0,2) -- (2,2); 
		\end{circuitikz}
	\end{center}
	
	\subsubsection{Current}
	Current in a series circuit is always the same.	In the case of the circuit above, 
	\[I_1 = I_2 = I_3\]
	
	\subsubsection{Voltage}
	The sum of voltages across components in a series circuit is equal to the voltage across the entire circuit.	In the case of the circuit above, 
	\[V_1 = V_2 + V_3\]
	
	\subsubsection{Resistance}
	\peqn{Resistance in Series}{If multiple resistors are arranged in series
		\begin{center}
			\begin{circuitikz}
				\draw (0,0) to[european resistor, l=\(R_1\)] (2,0) to[european resistor, l=\(R_2\)] (4,0) to[open] (5,0) to[european resistor, l=\(R_n\)] (7,0);
				\draw [dashed] (4,0) -- (5,0);
			\end{circuitikz}
		\end{center}
	then the net resistance is}{R_{\mathrm{net}} = R_1 + R_2 + \cdots R_n}

	\subsection{Parallel Circuits}	
	We will look at this series circuit for this subsection.
	\begin{center}
		\begin{circuitikz}
			\draw (2,2) to[battery1, i^=\(I_1\), v=\(V_1\)] (4,2) -- (6,2) -- (6,0) -- (4,0) to[european resistor, i<=\(I_2\), v=\(V_2\)] (2,0) -- (0,0) -- (0,2) -- (2,2); 
			\draw (6,0) -- (6,-2) -- (4,-2) to[european resistor, i<=\(I_3\), v=\(V_3\)] (2,-2) -- (0,-2) -- (0,0); 
		\end{circuitikz}
	\end{center}

	\subsubsection{Current}
	The sum of individual currents in each parallel branch is equal to the main current flowing into or out of parallel branches. In the case of this circuit,
	\[ I_1 = I_2 + I_3 \]
	
	\subsubsection{Voltage}
	The voltages across parallel branches are the same. In the case of this circuit,
	\[ V_1 = V_2 = V_3 \]
	
	\subsubsection{Resistance}
	\peqn{Resistance in Parallel}{If multiple resistors are arranged in parallel
		\begin{center}
			\begin{circuitikz}
				\draw (0,0) to[european resistor, l=\(R_1\)] (2,0) -- (2,-1);
				\draw (0,0) -- (0,-1) to[european resistor, l=\(R_2\)] (2,-1) -- (3,-1);
				\draw (-1,-1) -- (0,-1) -- (0,-1.25);
				\draw (0,-1.75) -- (0,-2);
				\draw [dashed] (0,-1.25) -- (0,-1.75);
				\draw (2,-1) -- (2,-1.25);
				\draw (2,-1.75) -- (2,-2);
				\draw [dashed] (2,-1.25) -- (2,-1.75);
				\draw (0,-2) to[european resistor, l=\(R_n\)] (2,-2);
			\end{circuitikz}
		\end{center}
		then the net resistance is}{R_{\mathrm{net}} = \left(R_1^{-1} + R_2^{-1} + \cdots + R_n^{-1}\right)^{-1}}
	
	\subsection{Voltage Divider}
	\begin{center}
		\begin{tikzpicture}
			\draw (1,5) -- (2,5 )to[battery1,v=\(E\)] (4,5) -- (5,5) --(5,3) to[european resistor, l=\(R_2\), v=\(V_2\)] (3,3) to[european resistor, l=\(R_1\), v=\(V_{\mathrm{out}}\)] (1,3) -- (1,5);
			\draw[-o] (1,3) -- (1,2);
			\draw[-o] (3,3) -- (3,2); 
			\draw[latex-latex] (1.1,1.8) -- (2.9,1.8) node[pos=0.5, anchor=north]{\(V_{\mathrm{out}}\)};
		\end{tikzpicture}
	\end{center}
	\[V_{\mathrm{out}} = \frac{R_1}{R_1 + R_2} \times E\]
	\peqn{Voltage Divider}{For a resistor \(R_x\) in a series circuit with total resistance \(R_T\), the voltage across the resistor \(R_x\) is}{V_x = \frac{R_x}{R_T} \times E}
	
	\subsection{Input and Output Transducers}
	\pdef{Input Transducer}{An input transducer is an electronic device that converts non-electrical energy into electrical energy.}
	We will look at two input transducers: (NTC-) thermistors and light dependent resistors (LDR).
	
	Thermistors are devices which vary its resistance according to temperature. As the temperature increases, the resistance decreases.
	
	\[
		R_{\mathrm{TH}} \begin{cases}
			\uparrow & T \downarrow \\
			\downarrow & T \uparrow
		\end{cases}
	\]
	
	Light-dependent resistors (LDR) varies its resistance according to the light intensity shining on it. As the light intensity shining on it increases, the resistance decreases.
	\[
		R_{\mathrm{LDR}} \begin{cases}
			\uparrow & \text{light intensity} \downarrow \\
			\downarrow & \text{light intensity} \uparrow
		\end{cases}
	\]
\end{document}
