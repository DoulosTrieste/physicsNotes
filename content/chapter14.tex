\documentclass[../main.tex]{subfiles}

\begin{document}
	\section{Electromagnetic Spectrum}
	\begin{preamb}
		The electromagnetic spectrum consists of electromagnetic waves of different frequencies. In this chapter we will explore these different frequencies and study some of their uses.
	\end{preamb}

	\subsection{Electromagnetic Waves}
	\pdef{Speed of Light}{All electromagnetic waves travel at the speed of light \(c\) in a vacuum. \[c = \SI{3.0e8}{\meter\per\second}\]}
	Some properties of electromagnetic waves:
	\begin{itemize}
		\item They do not require a medium to travel.
		\item They transfer energy from one place to another.
		\item They obey the laws of reflection and refraction.
		\item They do not change its frequency.
		\item They carry no electric charge.
	\end{itemize}
	
	\subsection{Parts of the Electromagnetic Spectrum}
	In increasing frequency (i.e. decreasing wavelength), and their uses:
	\begin{itemize}
		\item Radio waves (e.g. radio and television communication)
		\item Microwaves (e.g. microwave oven and satellite television)
		\item Infra-red (e.g. infra-red remote controllers and intruder alarms)
		\item Visible light (e.g. optical fibres for medical uses and telecommunications)
		\item Ultra-violet (e.g. sunbeds and sterilisation)
		\item X-rays (e.g. radiological and engineering applications)
		\item Gamma rays (e.g. medical treatment)
	\end{itemize}
	
	\subsection{Effects of the Electromagnetic Spectrum}
	When absorbing electromagnetic waves of various frequencies, different effects can be observed.
	\begin{itemize}
		\item Absorbing infrared rays can cause heating
		\item Higher frequencies such as x-rays can cause ionisation
		\item Overexposure to ultra-violet and higher frequency rays can lead to damage to living cells and tissue
	\end{itemize}
\end{document}
